\documentclass[11pt,a4paper]{article}

\usepackage[T1]{fontenc}
\usepackage[utf8]{inputenc}
\usepackage[polish]{babel}
\usepackage{lmodern}
\usepackage[final]{microtype}

\usepackage[a4paper,margin=2.5cm]{geometry}
\usepackage{fancyhdr}
\pagestyle{fancy}
\fancyhf{} % czyść wszystko
\lhead{Sprawozdanie -- Lista 2.}
\rhead{\leftmark}
\cfoot{\thepage}

\usepackage{amsmath,amssymb,amsthm,mathtools}
\usepackage{siunitx} % jeżeli potrzebne jednostki
\sisetup{locale = PL} % separator 1,23 (polski), można zmienić na locale=PL gdy dostępne
\usepackage{bm}       % pogrubione symbole

\usepackage[hidelinks]{hyperref}
\usepackage[nameinlink,capitalise,noabbrev]{cleveref}
\usepackage{csquotes}

\usepackage{enumitem}
\setlist{noitemsep,topsep=3pt}
\usepackage{xcolor}
\setlength{\headheight}{13.59999pt}

\title{Obliczenia Naukowe -- Sprawozdanie Laboratoria 2.}
\author{Jakub Kogut}
\date{10 listopada 2025}

\begin{document}
\maketitle

\section{Zadanie 1}
\subsection{Opis problemu}
W zadaniu należało powtórzyć eksperyment z zadania 5. z listy 1., jednakże ze zmienionymi danymi wejściowymi i zaobserwować ich wpływ.
\subsection{Rozwiązanie}
Wykorzystałem kod z poprzedniego zadania, zmieniając jedynie dane wejściowe, zgodnie z poleceniem: usunąłem ostatnią 9 z $x_4$ oraz ostatnią 7 z $x_5$.
\subsection{Wyniki}
Tabela przedstawia wyniki eksperymentu oraz porównanie z wynikami na starych danych.
\begin{table}[h!]
    \centering
    \begin{tabular}{|c|c|c|c|c|}
        \hline
        Algorytm & Float32 Stare & Float64 Stare & Float32 Nowe &Float64 Nowe\\
        \hline
        ``w przód'' & $-4.9994430 \cdot 10^{-1}$ & $1.025188136829 \cdot 10^{-10}$ &$-4.9994430 \cdot 10^{-1} $& $-4.296342739891\cdot 10^{-3}$\\
        \hline
        ``w tył'' & $-4.5434570 \cdot 10^{-1}$ & $-1.564330887049 \cdot 10^{-10}$ &$-4.5434570\cdot 10^{-1}$&$-4.296342998713\cdot 10^{-3}$\\
        \hline
        rosnąco\footnote{co do wartości bezwzględnej} & $-5.0000000 \cdot 10^{-1}$ & $0.000000000000 \cdot 10^{0}$ & $-5.0000000 \cdot 10^{-1}$&$-4.296342842280\cdot 10^{-3}$\\
        \hline
        malejąco & $-5.0000000 \cdot 10^{-1}$ & $0.000000000000\cdot 10^{0}$&$-5.0000000 \cdot 10^{-1}$&$-4.296342842280\cdot 10^{-3}$ \\
        \hline
    \end{tabular}
    \label{tab:iloczyn}
\end{table}
\subsection{Wnioski}
Można zauważyć że wyniki dla Float32 nie uległy zmianie, natomiast dla Float64 zmieniły się znacząco. Dzieje się tak dlatego, że w przypadku Float32 precyzja jest na tyle niska, że usunięcie tych elementów nie miało wpływu na wynik końcowy. W przypadku Float64 precyzja jest dużo większa, więc niewielka zmiana w danych wejściowych miała wpływ na wynik końcowy, po czym można wnioskować, że zadanie policzenia iloczynu skalarnego jest źle uwarunkowane.

\section{Zadanie 2}
\subsection{Opis problemu}
W zadaniu należy narysować wykres funkcji $f(x) = e^x \ln(1 + e^{-x})$ przy użyciu dwóch programów do wizualizacji. Następnie należy policzyć granicę tej funkcji przy $x \to \infty$ oraz porównać ją z wykresem.
\subsection{Rozwiązanie}
Do wizualizacji użyłem programu \textit{Desmos} oraz \textit{Wolfram Alpha}.
%zalacz obrazek wykresy/desmos.png
\begin{figure}[!ht]
    \centering
    \includegraphics[width=0.7\textwidth]{wykresy/desmos.png}
    \caption{Wykres funkcji $f(x) = e^x \ln(1 + e^{-x})$ w programie Desmos}
    \label{fig:desmos}
\end{figure}
%zalacz obrazek wykresy/wolfram.png
\begin{figure}[!ht]
    \centering
    \includegraphics[width=0.7\textwidth]{wykresy/wolfram.png}
    \caption{Wykres funkcji $f(x) = e^x \ln(1 + e^{-x})$ w programie Wolfram Alpha}
    \label{fig:wolfram}
\end{figure}
Obliczenie granicy:
\begin{align*}
    \lim_{x \to \infty} e^x \ln(1 + e^{-x}) &= \lim_{x \to \infty} \frac{\ln(1 + e^{-x})}{e^{-x}} \\
    &= \lim_{x \to \infty} \frac{\frac{d}{dx}(\ln(1 + e^{-x}))}{\frac{d}{dx}(e^{-x})} &\text{(Reguła de l'Hospitala)}\\
    &= \lim_{x \to \infty} \frac{\frac{-e^{-x}}{1 + e^{-x}}}{-e^{-x}} \\
    &= \lim_{x \to \infty} \frac{1}{1 + e^{-x}} = 1
\end{align*}
\subsection{Wnioski}
Na obydwu wykresach widać, że w okolicy $x \in [30, 36]$ funkcja zaczyna zachowywać się niestabilnie, a następnie dla większych wartości $x$ przyjmuje wartość 0, co jest sprzeczne z teorią, gdyż granica funkcji przy $x \to \infty$ wynosi 1. Dzieje się tak, ponieważ dla dużych wartości $x$ wyrażenie $e^{x}$ staje się bardzo duże, natomiast $\ln(1 + e^{-x})$ bardzo małe, przez co ich iloczyn, w zastosowanej przez programy arytmetyce, jest obarczony dużym błędem. Dla jeszcze większych wartości $x>36$ wartość $\ln(1 + e^{-x})\approx 0$ zatem cała funkcja przyjmuje wartość 0.

\section{Zadanie 3}
\subsection{Opis problemu}
W zadaniu należy porównać dwie metody rozwiązywania układów równań liniowych: metodę eliminacji Gaussa oraz poprzez wzięcie macierzy odwrotnej. Eksperyment należy przeprowadzić dla różnych rozmiarów macierzy Hilberta oraz losowej o rosnącym wskaźniku uwarunkowania.
\subsection{Rozwiązanie}
Do rozwiązania zadania użyłem podanych funkcji w \texttt{hilb.jt} oraz \texttt{matcond.jt} do generowania macierzy Hilberta oraz losowej o zadanym wskaźniku uwarunkowania. Następnie dla każdej z macierzy generowałem wektor $b = A \cdot x$, gdzie $x$ to wektor jednostkowy. Następnie rozwiązywałem układ równań $A x = b$ za pomocą obu metod i liczyłem błąd bezwzględny pomiędzy otrzymanym rozwiązaniem a wektorem $x$.
\subsection{Wyniki}
Poniżej w tabelach znajdują się wyniki eksperymentów dla macierzy Hilberta oraz losowych.
\begin{table}[!ht]
    \centering
    \begin{tabular}{|c|c|c|c|c|c|}
        \hline
        Rozmiar & Wskaźnik uwarunkowania &Rząd& Błąd metody Gaussa & Błąd metody odwrotnej \\
        \hline
        1 & $1.0$ & 1 & $0.0$ & $0.0$ \\
        \hline
        2 & $19.28$ & 2 & $5.66 \cdot 10^{-16}$ & $1.40 \cdot 10^{-15}$ \\
        \hline
        3 & $524.06$ & 3 & $8.02 \cdot 10^{-15}$ & $0.0$ \\
        \hline
        4 & $1.55 \cdot 10^{4}$ & 4 & $4.14 \cdot 10^{-14}$ & $0.0$ \\
        \hline
        5 & $4.77 \cdot 10^{5}$ & 5 & $1.68 \cdot 10^{-12}$ & $3.35 \cdot 10^{-12}$ \\
        \hline
        6 & $1.50 \cdot 10^{7}$ & 6 & $2.62 \cdot 10^{-10}$ & $2.01 \cdot 10^{-10}$ \\
        \hline
        7 & $4.75 \cdot 10^{8}$ & 7 & $1.26 \cdot 10^{-8}$ & $4.71 \cdot 10^{-9}$ \\
        \hline
        8 & $1.53 \cdot 10^{10}$ & 8 & $6.12 \cdot 10^{-8}$ & $3.07 \cdot 10^{-7}$ \\
        \hline
        9 & $4.93 \cdot 10^{11}$ & 9 & $3.87 \cdot 10^{-6}$ & $4.54 \cdot 10^{-6}$ \\
        \hline
        10 & $1.60 \cdot 10^{13}$ & 10 & $8.67 \cdot 10^{-5}$ & $2.50 \cdot 10^{-4}$ \\
        \hline
        11 & $5.22 \cdot 10^{14}$ & 10 & $1.58 \cdot 10^{-4}$ & $7.61 \cdot 10^{-3}$ \\
        \hline
        12 & $1.75 \cdot 10^{16}$ & 11 & $1.34 \cdot 10^{-1}$ & $2.59 \cdot 10^{-1}$ \\
        \hline
        13 & $3.34 \cdot 10^{18}$ & 11 & $1.10 \cdot 10^{-1}$ & $5.33$ \\
        \hline
        14 & $6.20 \cdot 10^{17}$ & 11 & $1.46$ & $8.71$ \\
        \hline
        15 & $3.67 \cdot 10^{17}$ & 12 & $4.70$ & $7.34$ \\
        \hline
        16 & $7.87 \cdot 10^{17}$ & 12 & $5.42 \cdot 10^{1}$ & $2.98 \cdot 10^{1}$ \\
        \hline
        17 & $1.26 \cdot 10^{18}$ & 12 & $1.37 \cdot 10^{1}$ & $1.05 \cdot 10^{1}$ \\
        \hline
        18 & $2.24 \cdot 10^{18}$ & 12 & $1.03 \cdot 10^{1}$ & $2.48 \cdot 10^{1}$ \\
        \hline
    \end{tabular}
    \caption{Wyniki eksperymentu dla macierzy Hilberta}
    \label{tab:hilbert}
\end{table}
\begin{table}[!ht]
    \centering
    \begin{tabular}{|c|c|c|c|c|c|}
        \hline
        Rozmiar & Wskaźnik uwarunkowania &Rząd& Błąd metody Gaussa & Błąd metody odwrotnej \\
        \hline
        5 & $1.0$ & 5 & $2.48 \cdot 10^{-16}$ & $1.11 \cdot 10^{-16}$ \\
        \hline
        5 & $10.0$ & 5 & $3.29 \cdot 10^{-16}$ & $3.58 \cdot 10^{-16}$ \\
        \hline
        5 & $1.00 \cdot 10^{3}$ & 5 & $5.74 \cdot 10^{-15}$ & $4.56 \cdot 10^{-15}$ \\
        \hline
        5 & $1.00 \cdot 10^{7}$ & 5 & $2.51 \cdot 10^{-10}$ & $2.31 \cdot 10^{-10}$ \\
        \hline
        5 & $1.00 \cdot 10^{12}$ & 5 & $6.32 \cdot 10^{-6}$ & $2.92 \cdot 10^{-6}$ \\
        \hline
        5 & $1.59 \cdot 10^{16}$ & 4 & $5.42 \cdot 10^{-1}$ & $2.76 \cdot 10^{-1}$ \\
        \hline
        10 & $1.0$ & 10 & $2.36 \cdot 10^{-16}$ & $1.57 \cdot 10^{-16}$ \\
        \hline
        10 & $10.0$ & 10 & $2.87 \cdot 10^{-16}$ & $2.16 \cdot 10^{-16}$ \\
        \hline
        10 & $1.00 \cdot 10^{3}$ & 10 & $1.55 \cdot 10^{-15}$ & $2.73 \cdot 10^{-15}$ \\
        \hline
        10 & $1.00 \cdot 10^{7}$ & 10 & $5.40 \cdot 10^{-10}$ & $5.70 \cdot 10^{-10}$ \\
        \hline
        10 & $1.00 \cdot 10^{12}$ & 10 & $1.32 \cdot 10^{-5}$ & $4.48 \cdot 10^{-6}$ \\
        \hline
        10 & $2.43 \cdot 10^{16}$ & 9 & $2.28 \cdot 10^{-1}$ & $2.24 \cdot 10^{-1}$ \\
        \hline
        20 & $1.0$ & 20 & $5.43 \cdot 10^{-16}$ & $6.32 \cdot 10^{-16}$ \\
        \hline
        20 & $10.0$ & 20 & $5.42 \cdot 10^{-16}$ & $6.07 \cdot 10^{-16}$ \\
        \hline
        20 & $1.00 \cdot 10^{3}$ & 20 & $2.15 \cdot 10^{-14}$ & $1.99 \cdot 10^{-14}$ \\
        \hline
        20 & $1.00 \cdot 10^{7}$ & 20 & $5.62 \cdot 10^{-10}$ & $5.39 \cdot 10^{-10}$ \\
        \hline
        20 & $1.00 \cdot 10^{12}$ & 20 & $5.07 \cdot 10^{-5}$ & $5.19 \cdot 10^{-5}$ \\
        \hline
        20 & $3.24 \cdot 10^{16}$ & 19 & $1.72 \cdot 10^{-1}$ & $1.44 \cdot 10^{-1}$ \\
        \hline
    \end{tabular}
    \caption{Wyniki eksperymentu dla macierzy losowych}
    \label{tab:random}
\end{table}

\subsection{Wnioski}
Dla wysokiego wskaźnika uwarunkowania, nie zależnie od rozmiuaru macierzy, obie metody dają duże błędy, jednak metoda eliminacji Gaussa radzi sobie lepiej niż metoda z macierzą odwrotną. Natomiast próbując rozwiązać układ równań dla macierzy Hilberta zwiększając jej rozmiar, błąd rośnie bardzo szybko, co pokazuje, że macierze Hilberta są źle uwarunkowane. 

\section{Zadanie 4}
\subsection{Opis problemu}
W zadaniu należy sprawdzić zachowanie programu obliczającego zera wielomianu Wilkinsona poprzez obliczenie jego zer dla postaci iloczynowej oraz rozwiniętej. Następnie zabużyć dane wejściowe i ponownie obliczyć zera wielomianu.
\subsection{Rozwiązanie}
Do policzenia zer wielomianu Wilkinsona użyłem funkcji \texttt{roots}. Następnie na ich podstawie obliczyłem wartość wielomianu w postaci iloczynowej jak i rozwiniętej. Powtarzam eksperyment dla zaburzonych danych: $a_{19} \rightarrow a_{19} + 2^{-23}$.
\subsection{Wyniki}
Poniższa tabela przedstawia wyniki eksperymentu. $p(x)$ to wartość wielomianu w postaci rozwiniętej, natomiast $P(x)$ to wartość wielomianu w postaci iloczynowej.
\begin{table}[!ht]
    \centering
    \begin{tabular}{|c|c|c|c|c|}
        \hline
        $k$ & $z_k$ & $|z_k - k|$ & $p(z_k)$ & $P(z_k)$ \\
        \hline
        1 & 1.00000 & $3.01092 \cdot 10^{-13}$ & $3.56965 \cdot 10^{4}$ & $3.66264 \cdot 10^{4}$ \\
        \hline
        2 & 2.00000 & $2.83182 \cdot 10^{-11}$ & $1.76253 \cdot 10^{5}$ & $1.81304 \cdot 10^{5}$ \\
        \hline
        3 & 3.00000 & $4.07903 \cdot 10^{-10}$ & $2.79158 \cdot 10^{5}$ & $2.90172 \cdot 10^{5}$ \\
        \hline
        4 & 4.00000 & $1.62625 \cdot 10^{-8}$ & $3.02711 \cdot 10^{6}$ & $2.04154 \cdot 10^{6}$ \\
        \hline
        5 & 5.00000 & $6.65770 \cdot 10^{-7}$ & $2.29175 \cdot 10^{7}$ & $2.08946 \cdot 10^{7}$ \\
        \hline
        6 & 5.99999 & $1.07542 \cdot 10^{-5}$ & $1.29024 \cdot 10^{8}$ & $1.12505 \cdot 10^{8}$ \\
        \hline
        7 & 7.00010 & $1.02003 \cdot 10^{-4}$ & $4.80511 \cdot 10^{8}$ & $4.57291 \cdot 10^{8}$ \\
        \hline
        8 & 7.99936 & $6.44170 \cdot 10^{-4}$ & $1.63795 \cdot 10^{9}$ & $1.55565 \cdot 10^{9}$ \\
        \hline
        9 & 9.00292 & $2.91529 \cdot 10^{-3}$ & $4.87707 \cdot 10^{9}$ & $4.68782 \cdot 10^{9}$ \\
        \hline
        10 & 9.99041 & $9.58696 \cdot 10^{-3}$ & $1.36386 \cdot 10^{10}$ & $1.26346 \cdot 10^{10}$ \\
        \hline
        11 & 11.02502 & $2.50229 \cdot 10^{-2}$ & $3.58563 \cdot 10^{10}$ & $3.30013 \cdot 10^{10}$ \\
        \hline
        12 & 11.95328 & $4.67167 \cdot 10^{-2}$ & $7.53333 \cdot 10^{10}$ & $7.38853 \cdot 10^{10}$ \\
        \hline
        13 & 13.07431 & $7.43140 \cdot 10^{-2}$ & $1.96060 \cdot 10^{11}$ & $1.84762 \cdot 10^{11}$ \\
        \hline
        14 & 13.91476 & $8.52444 \cdot 10^{-2}$ & $3.57513 \cdot 10^{11}$ & $3.55143 \cdot 10^{11}$ \\
        \hline
        15 & 15.07549 & $7.54938 \cdot 10^{-2}$ & $8.21627 \cdot 10^{11}$ & $8.42320 \cdot 10^{11}$ \\
        \hline
        16 & 15.94629 & $5.37133 \cdot 10^{-2}$ & $1.55150 \cdot 10^{12}$ & $1.57073 \cdot 10^{12}$ \\
        \hline
        17 & 17.02543 & $2.54271 \cdot 10^{-2}$ & $3.69474 \cdot 10^{12}$ & $3.31698 \cdot 10^{12}$ \\
        \hline
        18 & 17.99092 & $9.07865 \cdot 10^{-3}$ & $7.65011 \cdot 10^{12}$ & $6.34485 \cdot 10^{12}$ \\
        \hline
        19 & 19.00191 & $1.90982 \cdot 10^{-3}$ & $1.14353 \cdot 10^{13}$ & $1.22857 \cdot 10^{13}$ \\
        \hline
        20 & 19.99981 & $1.90709 \cdot 10^{-4}$ & $2.79241 \cdot 10^{13}$ & $2.31831 \cdot 10^{13}$ \\
        \hline
    \end{tabular}
    \caption{Wyniki eksperymentu dla wielomianu Wilkinsona dla niezaburzonych danych}
    \label{tab:wilk}
\end{table}
\begin{table}[!ht]
    \centering
    \begin{tabular}{|c|c|c|c|c|}
\hline
$k$ & $z_k$ & $|z_k-k|$ & $p(z_k)$ & $P(z_k)$\\
\hline
1  & $1.00000$ & $1.96287\cdot 10^{-13}$ & $2.44281\cdot 10^{4}$ & $2.38774\cdot 10^{4}$\\
\hline
2  & $2.00000$ & $1.47216\cdot 10^{-12}$ & $6.31468\cdot 10^{3}$ & $9.42529\cdot 10^{3}$\\
\hline
3  & $3.00000$ & $4.71376\cdot 10^{-10}$ & $4.21927\cdot 10^{5}$ & $3.35325\cdot 10^{5}$\\
\hline
4  & $4.00000$ & $1.19526\cdot 10^{-8}$  & $2.21907\cdot 10^{6}$ & $1.50048\cdot 10^{6}$\\
\hline
5  & $5.00000$ & $1.66405\cdot 10^{-7}$  & $6.22740\cdot 10^{6}$ & $5.22248\cdot 10^{6}$\\
\hline
6  & $5.99999$ & $7.05052\cdot 10^{-6}$  & $6.47173\cdot 10^{6}$ & $7.37588\cdot 10^{7}$\\
\hline
7  & $7.00030$ & $3.00994\cdot 10^{-4}$  & $3.06399\cdot 10^{7}$ & $1.34920\cdot 10^{9}$\\
\hline
8  & $7.99303$ & $6.96666\cdot 10^{-3}$  & $1.27955\cdot 10^{7}$ & $1.68773\cdot 10^{10}$\\
\hline
9  & $9.14748$ & $1.47477\cdot 10^{-1}$  & $7.24061\cdot 10^{7}$ & $2.19467\cdot 10^{11}$\\
\hline
10 & $9.50162$ & $4.98377\cdot 10^{-1}$  & $1.18200\cdot 10^{9}$ & $4.51692\cdot 10^{11}$\\
\hline
11 & $10.95346$ & $4.65447\cdot 10^{-2}$ & $6.78876\cdot 10^{12}$ & $6.08024\cdot 10^{10}$\\
\hline
12 & $10.95346$ & $1.04654\cdot 10^{0}$ & $6.78876\cdot 10^{12}$ & $6.08024\cdot 10^{10}$\\
\hline
13 & $12.99643$ & $3.56977\cdot 10^{-3}$ & $1.73366\cdot 10^{14}$ & $8.60218\cdot 10^{9}$\\
\hline
14 & $12.99643$ & $1.00357\cdot 10^{0}$ & $1.73366\cdot 10^{14}$ & $8.60218\cdot 10^{9}$\\
\hline
15 & $15.55554$ & $5.55539\cdot 10^{-1}$ & $5.26786\cdot 10^{15}$ & $5.84171\cdot 10^{12}$\\
\hline
16 & $15.55554$ & $4.44461\cdot 10^{-1}$ & $5.26786\cdot 10^{15}$ & $5.84171\cdot 10^{12}$\\
\hline
17 & $18.35914$ & $1.35914\cdot 10^{0}$ & $1.23311\cdot 10^{17}$ & $4.23334\cdot 10^{14}$\\
\hline
18 & $18.35914$ & $3.59140\cdot 10^{-1}$ & $1.23311\cdot 10^{17}$ & $4.23334\cdot 10^{14}$\\
\hline
19 & $20.50312$ & $1.50312\cdot 10^{0}$ & $1.31182\cdot 10^{18}$ & $3.09764\cdot 10^{17}$\\
\hline
20 & $20.50312$ & $5.03121\cdot 10^{-1}$ & $1.31182\cdot 10^{18}$ & $3.09764\cdot 10^{17}$\\
\hline
\end{tabular}

\caption{Wyniki eksperymentu dla wielomianu Wilkinsona dla zaburzonych danych\footnote{W miejscach gdzie pojawiają się liczby urojone biorę ich moduł}}
\end{table}

\subsection{Wnioski}
Na podstawie wyników zaburzonych można stwierdzić, że zadanie wyznaczania pierwiastków wielomianu Wilkinsona jest źle uwarunkowane, ponieważ nawet niewielka zmiana współczynnika powoduje duże zmiany w pierwiastkach oraz wartościach wielomianu w tych pierwiastkach. Dodatkowo można zauważyć, że wartości wielomianu w wyliczonym zerze znacznie różnią się od teoretycznej wartości 0, co wynika z braku precyzji arytmetyki Float64. Nie jesteśmy w stanie dokładnie przechować współczynników wielomianu, co powoduje błędy w obliczeniach.

\section{Zadanie 5}
\subsection{Opis problemu}
Mamy podane pewne równanie rekurencyjne opisujące rozmiar populacji w zależności od stanu poprzedniego pokolenia:
\[
    p_{n+1}= p_n + r p_n (1 - p_n)
\]
Należy przeprowadzić eksperyment porównujący zachowanie się równania dla 40 iteracji w artymetyce Float32 oraz Float64. Następnie w trakcie obliczeń należy wprowadzić niewielkie zaburzenie do wartości $p_{10}$ i ponownie wznowić obliczenia.
\subsection{Rozwiązanie}
Do rozwiązania zadania napisałem prostą pętlę iterującą 40 razy po równaniu rekurencyjnym.
\subsection{Wyniki}
%i, pF32, pF32 stopped, pF64
%1, 0.0397000, 0.0397000, 0.039700000000000
%2, 0.1540717, 0.1540717, 0.154071730000000
%3, 0.5450726, 0.5450726, 0.545072626044421
%4, 1.2889781, 1.2889781, 1.288978001188801
%5, 0.1715188, 0.1715188, 0.171519142109176
%6, 0.5978191, 0.5978191, 0.597820120107099
%7, 1.3191134, 1.3191134, 1.319113792413797
%8, 0.0562732, 0.0562732, 0.056271577646257
%9, 0.2155929, 0.2155929, 0.215586839232630
%10, 0.7229306, 0.7229306, 0.722914301179573
%11, 1.3238364, 1.3241479, 1.323841944168441
%12, 0.0377170, 0.0364884, 0.037695297254732
%13, 0.1466002, 0.1419594, 0.146518382713559
%14, 0.5219260, 0.5073804, 0.521670621435246
%15, 1.2704837, 1.2572169, 1.270261773935077
%16, 0.2395482, 0.2870845, 0.240352172778243
%17, 0.7860428, 0.9010855, 0.788101190235304
%18, 1.2905813, 1.1684768, 1.289094302790307
%19, 0.1655247, 0.5778930, 0.171084846701943
%20, 0.5799036, 1.3096911, 0.596529312494691
%21, 1.3107498, 0.0928922, 1.318575587982598
%22, 0.0888042, 0.3456818, 0.058377608259431
%23, 0.3315584, 1.0242395, 0.223286597599448
%24, 0.9964407, 0.9497582, 0.743575676395179
%25, 1.0070806, 1.0929108, 1.315588346001072
%26, 0.9856885, 0.7882812, 0.070035295602779
%27, 1.0280086, 1.2889631, 0.265426354520610
%28, 0.9416294, 0.1715748, 0.850351969060138
%29, 1.1065198, 0.5979856, 1.232112462387190
%30, 0.7529209, 1.3191822, 0.374146489639287
%31, 1.3110139, 0.0560039, 1.076629171428944
%32, 0.0877831, 0.2146064, 0.829125567400452
%33, 0.3280148, 0.7202578, 1.254154650050444
%34, 0.9892781, 1.3247173, 0.297906941472321
%35, 1.0210990, 0.0342414, 0.925382128557105
%36, 0.9564666, 0.1334483, 1.132532262669786
%37, 1.0813814, 0.4803680, 0.682241072715310
%38, 0.8173683, 1.2292118, 1.332605646962029
%39, 1.2652004, 0.3839622, 0.002909156902851
%40, 0.2586055, 1.0935680, 0.011611238029749
\begin{table}[!ht]
    \centering
    \begin{tabular}{|c|c|c|c|}
        \hline
        Iteracja & $p_n$ Float32 & $p_n$ Float32 po zaburzeniu & $p_n$ Float64 \\
        \hline
        1 & 0.0397000 & 0.0397000 & 0.039700000000000 \\
        \hline
        2 & 0.1540717 & 0.1540717 & 0.154071730000000 \\
        \hline
        3 & 0.5450726 & 0.5450726 & 0.545072626044421 \\
        \hline
        4 & 1.2889781 & 1.2889781 & 1.288978001188801 \\
        \hline
        5 & 0.1715188 & 0.1715188 & 0.171519142109176 \\
        \hline
        6 & 0.5978191 & 0.5978191 & 0.597820120107099 \\
        \hline
        7 & 1.3191134 & 1.3191134 & 1.319113792413797 \\
        \hline
        8 & 0.0562732 & 0.0562732 & 0.056271577646257 \\
        \hline
        9 & 0.2155929 & 0.2155929 & 0.215586839232630 \\
        \hline
        10 & 0.7229306 & 0.7229306 & 0.722914301179573 \\
        \hline
        11 & 1.3238364 & 1.3241479 & 1.323841944168441 \\
        \hline
        12 & 0.0377170 & 0.0364884 & 0.037695297254732 \\
        \hline
        13 & 0.1466002 & 0.1419594 & 0.146518382713559 \\
        \hline
        14 & 0.5219260 & 0.5073804 & 0.521670621435246 \\
        \hline
        15 & 1.2704837 & 1.2572169 & 1.270261773935077 \\
        \hline
        16 & 0.2395482 & 0.2870845 & 0.240352172778243 \\
        \hline
        17 & 0.7860428 & 0.9010855 & 0.788101190235304 \\
        \hline
        18 & 1.2905813 & 1.1684768 & 1.289094302790307 \\
        \hline
        19 & 0.1655247 & 0.5778930 & 0.171084846701943 \\
        \hline
        20 & 0.5799036 & 1.3096911 & 0.596529312494691 \\
        \hline
        21 & 1.3107498 & 0.0928922 & 1.318575587982598 \\
        \hline
        22 & 0.0888042 & 0.3456818 & 0.058377608259431 \\
        \hline
        23 & 0.3315584 & 1.0242395 & 0.223286597599448 \\
        \hline
        24 & 0.9964407 & 0.9497582 & 0.743575676395179 \\
        \hline
        25 & 1.0070806 & 1.0929108 & 1.315588346001072 \\
        \hline
        26 & 0.9856885 & 0.7882812 & 0.070035295602779 \\
        \hline
        27 & 1.0280086 & 1.2889631 & 0.265426354520610 \\
        \hline
        28 & 0.9416294 & 0.1715748 & 0.850351969060138 \\
        \hline
        29 & 1.1065198 & 0.5979856 & 1.232112462387190 \\
        \hline
        30 & 0.7529209 & 1.3191822 & 0.374146489639287 \\
        \hline
        31 & 1.3110139 & 0.0560039 & 1.076629171428944 \\
        \hline
        32 & 0.0877831 & 0.2146064 & 0.829125567400452 \\
        \hline
        33 & 0.3280148 & 0.7202578 & 1.254154650050444 \\
        \hline
        34 & 0.9892781 & 1.3247173 & 0.297906941472321 \\
        \hline
        35 & 1.0210990 & 0.0342414 & 0.925382128557105 \\
        \hline
        36 & 0.9564666 & 0.1334483 & 1.132532262669786 \\
        \hline
        37 & 1.0813814 & 0.4803680 & 0.682241072715310 \\
        \hline
        38 & 0.8173683 & 1.2292118 & 1.332605646962029 \\
        \hline
        39 & 1.2652004 & 0.3839622 & 0.002909156902851 \\
        \hline
        40 & 0.2586055 & 1.0935680 & 0.011611238029749 \\
        \hline
    \end{tabular}
    \caption{Wyniki eksperymentu dla równania rekurencyjnego}
    \label{tab:rekur}
\end{table}
\subsection{Wnioski}
Widać, że taki rekurencyjny proces jest bardzo wrażliwy na zaburzenia, ponieważ nawet niewielka zmiana wartości w $p_{10}$ powoduje ogromne różnice w dalszych iteracjach. Przez kilka pierwszych iteracji (około 13) wyniki dla obu arytmetyk są podobne, jednak z czasem różnice stają się coraz większe. Aby poprawnie przprowadzić podniesienie do kwadratu należy przechowywać dwa razy więcej cyfr znaczących niż wynosi ilość cyfr liczby podnoszonej. Generalnie, cała ta procedura jest bardzo niestabilna numerycznie.

\section{Zadanie 6}
\subsection{Opis problemu}
Podobnie jak w porzednim zadaniu, mamy sprawdzić zachowanie się równiania rekurencyjnego:
\[
    x_{n+1}=x_n^2 + c
\]
dla różnych wartości parametru $c$ oraz różnych wartości początkowych $x_0$. Należy przeprowadzić eksperyment dla arytmetyki Float64.
\subsection{Rozwiązanie}
Poniżej tabela z wynikami eksperymentu.
% Tabela 1: c=-2,x0=1 ; c=2,x0=2 ; c=-2,x0≈2
\begin{table}[!ht]
    \centering
\begin{tabular}{|c|c|c|c|}
\hline
$n$ &
$x_n$ ($c=-2,\,x_0=1$) &
$x_n$ ($c=2,\,x_0=2$) &
$x_n$ ($c=-2,\,x_0=1.99999999999999$) \\
\hline
1  & $-1.000000$ & $6.000000$ & $2.000000$\\ \hline
2  & $-1.000000$ & $3.800000\cdot 10^{1}$ & $2.000000$\\ \hline
3  & $-1.000000$ & $1.446000\cdot 10^{3}$ & $2.000000$\\ \hline
4  & $-1.000000$ & $2.090918\cdot 10^{6}$ & $2.000000$\\ \hline
5  & $-1.000000$ & $4.371938\cdot 10^{12}$ & $2.000000$\\ \hline
6  & $-1.000000$ & $1.911384\cdot 10^{25}$ & $2.000000$\\ \hline
7  & $-1.000000$ & $3.653390\cdot 10^{50}$ & $2.000000$\\ \hline
8  & $-1.000000$ & $1.334726\cdot 10^{101}$ & $2.000000$\\ \hline
9  & $-1.000000$ & $1.781493\cdot 10^{202}$ & $2.000000$\\ \hline
10 & $-1.000000$ & $\infty$ & $2.000000$\\ \hline
11 & $-1.000000$ & $\infty$ & $2.000000$\\ \hline
12 & $-1.000000$ & $\infty$ & $2.000000$\\ \hline
13 & $-1.000000$ & $\infty$ & $1.999999$\\ \hline
14 & $-1.000000$ & $\infty$ & $1.999997$\\ \hline
15 & $-1.000000$ & $\infty$ & $1.999989$\\ \hline
16 & $-1.000000$ & $\infty$ & $1.999957$\\ \hline
17 & $-1.000000$ & $\infty$ & $1.999828$\\ \hline
18 & $-1.000000$ & $\infty$ & $1.999313$\\ \hline
19 & $-1.000000$ & $\infty$ & $1.997254$\\ \hline
20 & $-1.000000$ & $\infty$ & $1.989024$\\ \hline
21 & $-1.000000$ & $\infty$ & $1.956215$\\ \hline
22 & $-1.000000$ & $\infty$ & $1.826779$\\ \hline
23 & $-1.000000$ & $\infty$ & $1.337120$\\ \hline
24 & $-1.000000$ & $\infty$ & $-2.121097\cdot 10^{-1}$\\ \hline
25 & $-1.000000$ & $\infty$ & $-1.955009$\\ \hline
26 & $-1.000000$ & $\infty$ & $1.822062$\\ \hline
27 & $-1.000000$ & $\infty$ & $1.319910$\\ \hline
28 & $-1.000000$ & $\infty$ & $-2.578368\cdot 10^{-1}$\\ \hline
29 & $-1.000000$ & $\infty$ & $-1.933520$\\ \hline
30 & $-1.000000$ & $\infty$ & $1.738500$\\ \hline
31 & $-1.000000$ & $\infty$ & $1.022383$\\ \hline
32 & $-1.000000$ & $\infty$ & $-9.547330\cdot 10^{-1}$\\ \hline
33 & $-1.000000$ & $\infty$ & $-1.088485$\\ \hline
34 & $-1.000000$ & $\infty$ & $-8.152007\cdot 10^{-1}$\\ \hline
35 & $-1.000000$ & $\infty$ & $-1.335448$\\ \hline
36 & $-1.000000$ & $\infty$ & $-2.165791\cdot 10^{-1}$\\ \hline
37 & $-1.000000$ & $\infty$ & $-1.953094$\\ \hline
38 & $-1.000000$ & $\infty$ & $1.814574$\\ \hline
39 & $-1.000000$ & $\infty$ & $1.292680$\\ \hline
40 & $-1.000000$ & $\infty$ & $-3.289791\cdot 10^{-1}$\\
\hline
\end{tabular}
    \caption{Wyniki eksperymentu dla równania rekurencyjnego z różnymi parametrami}
    \label{tab:rekur2}
\end{table}
\begin{table}[!ht]
    \centering
% Tabela 2: przypadki c=-1
\begin{tabular}{|c|c|c|c|c|}
\hline
$n$ &
$x_n$ ($c=-1,\,x_0=1$) &
$x_n$ ($c=-1,\,x_0=-1$) &
$x_n$ ($c=-1,\,x_0=0.75$) &
$x_n$ ($c=-1,\,x_0=0.25$) \\
\hline
1  & $0.000000$ & $0.000000$ & $-4.375000\cdot 10^{-1}$ & $-9.375000\cdot 10^{-1}$\\ \hline
2  & $-1.000000$ & $-1.000000$ & $-8.085938\cdot 10^{-1}$ & $-1.210938\cdot 10^{-1}$\\ \hline
3  & $0.000000$ & $0.000000$ & $-3.461761\cdot 10^{-1}$ & $-9.853363\cdot 10^{-1}$\\ \hline
4  & $-1.000000$ & $-1.000000$ & $-8.801621\cdot 10^{-1}$ & $-2.911237\cdot 10^{-2}$\\ \hline
5  & $0.000000$ & $0.000000$ & $-2.253147\cdot 10^{-1}$ & $-9.991525\cdot 10^{-1}$\\ \hline
6  & $-1.000000$ & $-1.000000$ & $-9.492333\cdot 10^{-1}$ & $-1.694342\cdot 10^{-3}$\\ \hline
7  & $0.000000$ & $0.000000$ & $-9.895619\cdot 10^{-2}$ & $-9.999971\cdot 10^{-1}$\\ \hline
8  & $-1.000000$ & $-1.000000$ & $-9.902077\cdot 10^{-1}$ & $-5.741579\cdot 10^{-6}$\\ \hline
9  & $0.000000$ & $0.000000$ & $-1.948876\cdot 10^{-2}$ & $-1.000000$\\ \hline
10 & $-1.000000$ & $-1.000000$ & $-9.996202\cdot 10^{-1}$ & $-6.593148\cdot 10^{-11}$\\ \hline
11 & $0.000000$ & $0.000000$ & $-7.594796\cdot 10^{-4}$ & $-1.000000$\\ \hline
12 & $-1.000000$ & $-1.000000$ & $-9.999994\cdot 10^{-1}$ & $0.000000$\\ \hline
13 & $0.000000$ & $0.000000$ & $-1.153618\cdot 10^{-6}$ & $-1.000000$\\ \hline
14 & $-1.000000$ & $-1.000000$ & $-1.000000$ & $0.000000$\\ \hline
15 & $0.000000$ & $0.000000$ & $-2.661649\cdot 10^{-12}$ & $-1.000000$\\ \hline
16 & $-1.000000$ & $-1.000000$ & $-1.000000$ & $0.000000$\\ \hline
17 & $0.000000$ & $0.000000$ & $0.000000$ & $-1.000000$\\ \hline
18 & $-1.000000$ & $-1.000000$ & $-1.000000$ & $0.000000$\\ \hline
19 & $0.000000$ & $0.000000$ & $0.000000$ & $-1.000000$\\ \hline
20 & $-1.000000$ & $-1.000000$ & $-1.000000$ & $0.000000$\\ \hline
21 & $0.000000$ & $0.000000$ & $0.000000$ & $-1.000000$\\ \hline
22 & $-1.000000$ & $-1.000000$ & $-1.000000$ & $0.000000$\\ \hline
23 & $0.000000$ & $0.000000$ & $0.000000$ & $-1.000000$\\ \hline
24 & $-1.000000$ & $-1.000000$ & $-1.000000$ & $0.000000$\\ \hline
25 & $0.000000$ & $0.000000$ & $0.000000$ & $-1.000000$\\ \hline
26 & $-1.000000$ & $-1.000000$ & $-1.000000$ & $0.000000$\\ \hline
27 & $0.000000$ & $0.000000$ & $0.000000$ & $-1.000000$\\ \hline
28 & $-1.000000$ & $-1.000000$ & $-1.000000$ & $0.000000$\\ \hline
29 & $0.000000$ & $0.000000$ & $0.000000$ & $-1.000000$\\ \hline
30 & $-1.000000$ & $-1.000000$ & $-1.000000$ & $0.000000$\\ \hline
31 & $0.000000$ & $0.000000$ & $0.000000$ & $-1.000000$\\ \hline
32 & $-1.000000$ & $-1.000000$ & $-1.000000$ & $0.000000$\\ \hline
33 & $0.000000$ & $0.000000$ & $0.000000$ & $-1.000000$\\ \hline
34 & $-1.000000$ & $-1.000000$ & $-1.000000$ & $0.000000$\\ \hline
35 & $0.000000$ & $0.000000$ & $0.000000$ & $-1.000000$\\ \hline
36 & $-1.000000$ & $-1.000000$ & $-1.000000$ & $0.000000$\\ \hline
37 & $0.000000$ & $0.000000$ & $0.000000$ & $-1.000000$\\ \hline
38 & $-1.000000$ & $-1.000000$ & $-1.000000$ & $0.000000$\\ \hline
39 & $0.000000$ & $0.000000$ & $0.000000$ & $-1.000000$\\ \hline
40 & $-1.000000$ & $-1.000000$ & $-1.000000$ & $0.000000$\\
\hline
\end{tabular}
    \caption{Wyniki eksperymentu dla równania rekurencyjnego dla $c=-1$ i różnych wartości początkowych}
    \label{tab:rekur3}
\end{table}

\subsection{Wnioski}
Równanie rekurencyjne można zinterpretować graficznie:
%wstaw wykresy w wykresy/wykresN.png gdzie N od 1 do 7
\begin{figure}[!h]
    \centering
    \includegraphics[width=0.8\textwidth]{wykresy/wykres1.png}
    \caption{Wykres dla $c=-2$, $x_0=1$}
    \label{fig:wyk1}
\end{figure}
\begin{figure}[!h]
    \centering
    \includegraphics[width=0.8\textwidth]{wykresy/wykres2.png}
    \caption{Wykres dla $c=2$, $x_0=2$}
    \label{fig:wyk2}
\end{figure}
\begin{figure}[!h]
    \centering
    \includegraphics[width=0.8\textwidth]{wykresy/wykres3.png}
    \caption{Wykres dla $c=-2$, $x_0\approx2$}
    \label{fig:wyk3}
\end{figure}
\begin{figure}[!h]
    \centering
    \includegraphics[width=0.8\textwidth]{wykresy/wykres4.png}
    \caption{Wykres dla $c=-1$, $x_0=1$}
    \label{fig:wyk4}
\end{figure}
\begin{figure}[!h]
    \centering
    \includegraphics[width=0.8\textwidth]{wykresy/wykres5.png}
    \caption{Wykres dla $c=-1$, $x_0=-1$}
    \label{fig:wyk5}
\end{figure}
\begin{figure}[!h]
    \centering
    \includegraphics[width=0.8\textwidth]{wykresy/wykres6.png}
    \caption{Wykres dla $c=-1$, $x_0=0.75$}
    \label{fig:wyk6}
\end{figure}
\begin{figure}[!h]
    \centering
    \includegraphics[width=0.8\textwidth]{wykresy/wykres7.png}
    \caption{Wykres dla $c=-1$, $x_0=0.25$}
    \label{fig:wyk7}
\end{figure}

Widać, że dla różnych wartości parametru $c$ oraz różnych wartości początkowych $x_0$ zachowanie się równania rekurencyjnego jest zupełnie inne. Dla $c=-2$ i $x_0=1$ ciąg zbiega do stałej wartości -1. Dla $c=2$ i $x_0=2$ ciąg rośnie bardzo szybko i osiąga wartość nieskończoności już po 10 iteracjach. Dla $c=-2$ i $x_0$ bliskiego 2, ciąg pozostaje bliski wartości 2 przez wiele iteracji, ale ostatecznie zaczyna oscylować i oddalać się od tej wartości.\\
W niektórych przypadkach, jak dla $c=-1$ i $x_0=1$ lub $x_0=-1$, ciąg oscyluje między dwoma wartościami (-1 i 0). Dla $c=-1$ i $x_0=0.75$ lub $x_0=0.25$, po pewnej ilości iteracji ciąg również wpada w cykl oscylacji między -1 a 0.\\

\end{document}
