\documentclass[11pt,a4paper]{article}

\usepackage[T1]{fontenc}
\usepackage[utf8]{inputenc}
\usepackage[polish]{babel}
\usepackage{lmodern}
\usepackage[final]{microtype}

\usepackage[a4paper,margin=2.5cm]{geometry}
\usepackage{fancyhdr}
\pagestyle{fancy}
\fancyhf{} % czyść wszystko
\lhead{Notatki z matematyki}
\rhead{\leftmark}
\cfoot{\thepage}

\usepackage{amsmath,amssymb,amsthm,mathtools}
\usepackage{siunitx} % jeżeli potrzebne jednostki
\sisetup{locale = DE} % separator 1,23 (polski), można zmienić na locale=PL gdy dostępne
\usepackage{bm}       % pogrubione symbole

\usepackage[hidelinks]{hyperref}
\usepackage[nameinlink,capitalise,noabbrev]{cleveref}
\usepackage{csquotes}

\usepackage{enumitem}
\setlist{noitemsep,topsep=3pt}
\usepackage{xcolor}

\title{Obliczenia Naukowe -- Sprawozdanie Laboratoria 3.}
\author{Jakub Kogut}
\date{\today}

\begin{document}
\maketitle

\section{Wstęp}
Na 3. liscie zadań mamy za zadanie napisać oraz przetowstować różne metody rozwiązywania równiań $f(x) = 0$, gdzie $f$ jest zadaną funkcją rzeczywistą jednej zmiennej. 
\section{Metody iteracyjne}
\subsection{Metoda bisekcji}
Metoda bisekcji polega na podziale przedziału $[a,b]$, na którym funkcja $f$ zmienia znak, na pół i wyborze podprzedziału, na którym funkcja również zmienia znak. Proces ten iterujemy aż do uzyskania zadowalającej dokładności. Z twierdzenia Darboux wynika, że jeżeli $f$ jest funkcją ciągłą na przedziale $[a,b]$ i $f(a)f(b) < 0$, to istnieje co najmniej jedno miejsce zerowe $f$ w przedziale $(a,b)$.

%dodaj wykres bisekcji
\begin{figure}[h!]
    \centering
    \includegraphics[width=0.7\textwidth]{plots/bisection_iterations.png}
    \caption{Ilustracja metody bisekcji z wykładu. Watości początkowe $a = 1, b = 5$.}
    \label{fig:bisekcja}
\end{figure}

Algorytm działania metody wygląda następująco:
\begin{enumerate}
    \item wyznacz środek przedziału $c = \frac{a+b}{2}$
    \item należy zbadać w którym z przedziałów $[a,c]$ lub $[c,b]$ funkcja zmienia znak poprzez sprawdzenie wartości $f(a)f(c)$
    \item jeżeli $f(a)f(c) < 0$, to należy przyjąć $b \leftarrow c$, w przeciwnym wypadku $a \leftarrow c$
    \item powtarzamy kroki 1-3 aż do uzyskania zadowalającej dokładności
\end{enumerate}
Z wykładu wiemy, że metoda ta jest zbieżna liniowo i działa globalnie.

\subsection{Metoda Newtona}
Metoda Newtona polega na przybliżaniu miejsca zerowego funkcji $f$ poprzez kolejne styczne do wykresu funkcji. Wybieramy punkt startowy $x_0$ i obliczamy kolejne przybliżenia według wzoru:
\[x_{n+1} = x_n - \frac{f(x_n)}{f'(x_n)}\]
Proces ten powtarzamy aż do uzyskania zadowalającej dokładności.
\begin{figure}[h!]
    \centering
    \includegraphics[width=0.7\textwidth]{plots/newton_iterations.png}
    \caption{Ilustracja metody Newtona z wykładu. Wartość początkowa $x_0 = 4.5$.}
    \label{fig:newton}
\end{figure}
Jednym z założeń tej metody jest to, że musi ona być podwójnie różniczkowalna w otoczeniu miejsca zerowego. Zatem, jeżeli badamy funkcję $f$ na przedziale $[a,b]$ musi ona spełniać:
\[
    f \in C^2[a,b] 
\]
Z wykładu wiemy, że jest ona lokalnie zbieżna kwadratowo.

\subsection{Metoda siecznych}
Metoda siecznych jest podobna do metody Newtona, jednak zamiast korzystać z pochodnej funkcji, korzysta z przybliżenia pochodnej za pomocą dwóch ostatnich punktów. Wybieramy dwa punkty startowe $x_0$ i $x_1$ i obliczamy kolejne przybliżenia według wzoru:
\[x_{n+1} = x_n - f(x_n) \cdot \frac{x_n - x_{n-1}}{f(x_n) - f(x_{n-1)}}\]
Proces ten powtarzamy aż do uzyskania zadowalającej dokładności.
\begin{figure}[h!]
    \centering
    \includegraphics[width=0.7\textwidth]{plots/secant_iterations.png}
    \caption{Ilustracja metody siecznych z wykładu. Wartości początkowe $x_0 = 1, x_1 = 5$.}
    \label{fig:sieczne}
\end{figure}
Natomiast ta metoda jest również lokalnie zbieżna, ale z wykładnikiem zbieżności $\varphi = \frac{1 + \sqrt{5}}{2}$.

\newpage
\section{Zadanie 4}
W zadaniu 4 należało przetestować zachowania zaimplementowanych metod na funkcji:
\[
    f(x) = sin(x) - \frac{x^2}{4}
\]
Możemy łatwo policzyć jej pochodną, która będzie potrzebna w metodzie Newtona:
\[
    f'(x) = cos(x) - \frac{x}{2}
\]
Dla wszystkich metod dana jest ta sama dokładność $\epsilon = 10^{-5}, \delta = 10^{-5}$.
\begin{figure}[!ht]
    \centering
    \includegraphics[width=0.7\textwidth]{plots/zadanie3.png}
    \caption{Wykres funkcji: \textcolor{red}{$f(x) = sin(x) - \frac{x^2}{4}$}.}
    \label{fig:zad3}
\end{figure}

%Metoda Newtona:
%Pierwiastek: 1.933753779789742
%Wartosc funkcji w pierwiastku: -2.2423316314856834e-8
%Liczba iteracji: 4
%Ostatni blad: 0
%
%Metoda siecznych:
%Pierwiastek: 1.933753644474301
%Wartosc funkcji w pierwiastku: 1.564525129449379e-7
%Liczba iteracji: 4
%Ostatni blad: 0


\begin{table}[h!]
    \centering
    \begin{tabular}{|c|c|c|c|c|}
        \hline
        Metoda & Przybliżone $r$ & Wartość $f(r)$ & Liczba iteracji & Kod Błędu \\
        \hline
        Bisekcja & 1.9337539672851562 & $-2.7027680138402843\cdot 10{-7}$ & 16 & 0 \\
        \hline
        Newton & 1.933753779789742 & $-2.2423316314856834\cdot 10^{-8}$ & 4 & 0 \\
        \hline
        Sieczne & 1.933753644474301 & $1.564525129449379\cdot 10^{-7}$ & 4 & 0 \\
        \hline
    \end{tabular}
    \caption{Wyniki działania metod na funkcji $f(x) = sin(x) - \frac{x^2}{4}$.}
    \label{tab:results}
\end{table}

Jak widać każda z metod wyszukuje poprawnie miejsca zerowego funkcji z zadaną dokładnością. Najszybciej zbiega metoda Newtona oraz metoda siecznych, które potrzebują jedynie 4 iteracji do uzyskania zadowalającego wyniku. Wynika to z faktu, że obie te metody są lokalnie zbieżne kwadratowo (Newton) oraz z wykładnikiem zbieżności $\varphi = \frac{1 + \sqrt{5}}{2}$ (sieczne). Metoda bisekcji potrzebuje aż 16 iteracji, co jest zgodne z jej liniową zbieżnością. Wszystkie wartości początkowe zostały dobrze dobrane. Ponieważ metody siecznych oraz Newtona są metodami lokalnymi, a ich punkty startowe znajdowały się wystarczająco blisko rzeczywistego miejsca zerowego funkcji.
\newpage
\section{Zadanie 5}
W zadaniu 5 mamy wyznaczyć punkt przecięcia się dwóch funkcji:
\[
    f_1(x) = 3x
\]
\[
    f_2(x) = e^x
\]
Aby to zrobić, musimy rozwiązać równanie:
\[
    f(x) = f_1(x) - f_2(x) = 3x - e^x = 0
\]
Zobaczmy również wykres tej funkcji:
\begin{figure}[!ht]
    \centering
    \includegraphics[width=0.7\textwidth]{plots/zadanie4.png}
    \caption{Wykres funkcji: \textcolor{green}{$f(x) = 3x - e^x$}, \textcolor{red}{$f_1(x) = 3x$}, \textcolor{blue}{$f_2(x) = e^x$}.}
    \label{fig:zad4}
\end{figure}
Jak widać na wykresie, funkcja $f$ ma dwa miejsca zerowe: jedno w przedziale $[0,1]$ oraz drugie w przedziale $[1,2]$. Zatem aby znaleźć oba miejsca zerowe, zastosujemy metodę bisekcji na obu przedziałach.
%Miejsce przeciecia funkcji f1 i f2 to: 0.61907958984375
%Wartosc funkcji f1 w tym punkcie to: 1.85723876953125
%Wartosc funkcji f2 w tym punkcie to: 1.8572178527553258
%Wartosc sygnalu bledu to: 0
%Liczba iteracji potrzebnych do osiagniecia tego bledu to: 14
%Miejsce przeciecia funkcji f1 i f2 to: 1.51214599609375
%Wartosc funkcji f1 w tym punkcie to: 4.53643798828125
%Wartosc funkcji f2 w tym punkcie to: 4.536455571851486
%Wartosc sygnalu bledu to: 0
%Liczba iteracji potrzebnych do osiagniecia tego bledu to: 14

\begin{table}[h!]
    \centering
    \begin{tabular}{|c|c|c|c|c|}
        \hline
        Przedział & Przybliżone $r$ & Wartość $f_1(r)$ & Wartość $f_2(r)$ & Liczba iteracji \\
        \hline
        $[0,1]$ & 0.61907958984375 & 1.85723876953125 & 1.8572178527553258 & 14 \\
        \hline
        $[1,2]$ & 1.51214599609375 & 4.53643798828125 & 4.536455571851486 & 14 \\
        \hline
    \end{tabular}
    \caption{Wyniki działania metody bisekcji na funkcji $f(x) = 3x - e^x$ na dwóch przedziałach.}
    \label{tab:zad5}
\end{table}
Aby nie sprawdzać ilości miejsc zerowych funkcji na danym przedziale, moglibyśmy zmodyfikować metodę bisekcji tak, aby działała rekurencyjnie na podprzedziałach, aż do znalezienia wszystkich miejsc zerowych z zadaną dokładnością. Wtedy można by było uniknąć pułkapki z wyborem złych przedziałów startowych.

\section{Zadanie 6}
W zadaniu należy znaleść miejsca zerowe funkcji:
\[
    f_1(x) = e^{1-x} - 1
\]
oraz
\[
    f_2(x) = x e^{-x}
\]
przy użyciu wsytkich dotyczasowych metod iteracyjnych. Zobaczmy wykresy tych funkcji:
\begin{figure}[!ht]
    \centering
    \includegraphics[width=0.7\textwidth]{plots/zadanie5.png}
    \caption{Wykresy funkcji: \textcolor{blue}{$f_1(x) = e^{1-x} - 1$}, \textcolor{red}{$f_2(x) = x e^{-x}$}.}
    \label{fig:zad5}
\end{figure}\\
Również dokładność dla wszystkich metod wynosi $\epsilon = 10^{-5}, \delta = 10^{-5}$.
\subsection{Metoda bisekcji}
Dla funkcji $f_1$ uruchomiłem metodę na kilku przedziałach aby zbadać jej zachowanie:
%Miejsce zerowe:-100,200, 1.0000020265579224, Wartosc funkcji w miejscu zerowym: -2.026555868894775e-6, Iteracje: 25, Blad: 0
%Miejsce zerowe:0, 1.5, 1.0000019073486328, Wartosc funkcji w miejscu zerowym: -1.9073468138230965e-6, Iteracje: 18, Blad: 0
%Miejsce zerowe:0.9999, 1.0001, 1.0, Wartosc funkcji w miejscu zerowym: 0.0, Iteracje: 1, Blad: 0
\begin{table}[h!]
    \centering
    \begin{tabular}{|c|c|c|c|c|}
        \hline
        Przedział & Przybliżone $r$ & Wartość $f_1(r)$ & Liczba iteracji & Kod Błędu \\
        \hline
        $[-100,200]$ & 1.0000020265579224 & $-2.026555868894775\cdot 10^{-6}$ & 25 & 0 \\
        \hline
        $[0,1.5]$ & 1.0000019073486328 & $-1.9073468138230965\cdot 10^{-6}$ & 18 & 0 \\
        \hline
        $[0.9999,1.0001]$ & 1.0 & 0.0 & 1 & 0 \\
        \hline
    \end{tabular}
    \caption{Wyniki działania metody bisekcji na funkcji $f_1(x) = e^{1-x} - 1$.}
    \label{tab:zad6_f1_bisect}
\end{table}

Dla funkcji $f_2$ analogicznie:
%a = -100.0, b = 100.0
%Miejsce zerowe: 100.0, Wartosc funkcji w miejscu zerowym: 3.7200759760208363e-42, Iteracje: 0, Blad: 0
%a = -1.0, b = 1.0
%Miejsce zerowe: 0.0, Wartosc funkcji w miejscu zerowym: 0.0, Iteracje: 1, Blad: 0
%a = -0.005, b = 0.005
%Miejsce zerowe: 0.0, Wartosc funkcji w miejscu zerowym: 0.0, Iteracje: 1, Blad: 0
%a = -0.95, b = 1.5
%Miejsce zerowe: 3.0517578126132883e-6, Wartosc funkcji w miejscu zerowym: 3.0517484994017526e-6, Iteracje: 14, Blad: 0
\begin{table}[h!]
    \centering
    \begin{tabular}{|c|c|c|c|c|}
        \hline
        Przedział & Przybliżone $r$ & Wartość $f_2(r)$ & Liczba iteracji & Kod Błędu \\
        \hline
        $[-100,100]$ & 100.0 & $3.7200759760208363\cdot 10^{-42}$ & 0 & 0 \\
        \hline
        $[-0.95, 1.5]$ & $3.0517578126132883\cdot 10^{-6}$ & $3.0517484994017526\cdot 10^{-6}$ & 14 & 0 \\
        \hline
        $[-1,1]$ & 0.0 & 0.0 & 1 & 0 \\
        \hline
        $[-0.005,0.005]$ & 0.0 & 0.0 & 1 & 0 \\
        \hline
    \end{tabular}
    \caption{Wyniki działania metody bisekcji na funkcji $f_2(x) = x e^{-x}$.}
    \label{tab:zad6_f2_bisect}
\end{table}
\newpage
Przy nie odpowiednim doborze przedziału startowego metoda bisekcji może nie znaleźć miejsca zerowego (pierwszy wiersz tabeli \ref{tab:zad6_f2_bisect}). Dzieje się tak ponieważ funkcja $f_2$ bardzo szybko zbliża się do 0, które jest brane za rozwiązanie.\\
Jeżeli szukany pierwiastek jest środkiem któregoś z przedziałów startowych, metoda znajdzie go od razu (drugi i trzeci wiersz tabeli \ref{tab:zad6_f1_bisect} oraz \ref{tab:zad6_f2_bisect}).

\subsection{Metoda Newtona}
Dla funkcji $f_1$:
%Metoda Newtona dla f1 z x0 = 10.0:
%Miejsce zerowe: NaN, Wartosc funkcji w miejscu zerowym: NaN, Iteracje: 1000, Blad: 1
%Metoda Newtona dla f1 z x0 = 5.0:
%Miejsce zerowe: 0.9999996427095682, Wartosc funkcji w miejscu zerowym: 3.572904956339329e-7, Iteracje: 54, Blad: 0

%Metoda Newtona dla f1 z x0 = 2.0:
%Miejsce zerowe: 0.9999999810061002, Wartosc funkcji w miejscu zerowym: 1.8993900008368314e-8, Iteracje: 5, Blad: 0
%Metoda Newtona dla f1 z x0 = 1.0:
%Miejsce zerowe: 1.0, Wartosc funkcji w miejscu zerowym: 0.0, Iteracje: 0, Blad: 0
%Metoda Newtona dla f1 z x0 = 0.5:
%Miejsce zerowe: 0.9999999998878352, Wartosc funkcji w miejscu zerowym: 1.1216494399945987e-10, Iteracje: 4, Blad: 0
%Metoda Newtona dla f1 z x0 = 0.0:
%Miejsce zerowe: 0.9999984358892101, Wartosc funkcji w miejscu zerowym: 1.5641120130194253e-6, Iteracje: 4, Blad: 0
%Metoda Newtona dla f1 z x0 = -1.0:
%Miejsce zerowe: 0.9999999999700886, Wartosc funkcji w miejscu zerowym: 2.991140668484604e-11, Iteracje: 6, Blad: 0
%Metoda Newtona dla f1 z x0 = -100.0:
%Miejsce zerowe: 0.9999999998780821, Wartosc funkcji w miejscu zerowym: 1.2191803122618694e-10, Iteracje: 105, Blad: 0
\begin{table}[h!]
    \centering
    \begin{tabular}{|c|c|c|c|c|}
        \hline
        Punkt startowy & Przybliżone $r$ & Wartość $f_1(r)$ & Liczba iteracji & Kod Błędu \\
        \hline
        10.0 & NaN & NaN & 1000 & 1 \\
        \hline
        5.0 & 0.9999996427095682 & $3.572904956339329\cdot 10^{-7}$ & 54 & 0 \\
        \hline
        2.0 & 0.9999999810061002 & $1.8993900008368314\cdot 10^{-8}$ & 5 & 0 \\
        \hline
        1.0 & 1.0 & 0.0 & 0 & 0 \\
        \hline
        0.5 & 0.9999999998878352 & $1.1216494399945987\cdot 10^{-10}$ & 4 & 0 \\
        \hline
        0.0 & 0.9999984358892101 & $1.5641120130194253\cdot 10^{-6}$ & 4 & 0 \\
        \hline
        -1.0 & 0.9999999999700886 & $2.991140668484604\cdot 10^{-11}$ & 6 & 0 \\
        \hline
        -100.0 & 0.9999999998780821 & $1.2191803122618694\cdot 10^{-10}$ & 105 & 0 \\
        \hline
    \end{tabular}
    \caption{Wyniki działania metody Newtona na funkcji $f_1(x) = e^{1-x} - 1$.}
    \label{tab:zad6_f1_newton}
\end{table}
Metoda Newtona dla funkcji $f_1$ działa dobrze dla punktów startowych blisko miejsca zerowego. Jednak dla punktu startowego daleko od miejsca zerowego (pierwszy wiersz tabeli \ref{tab:zad6_f1_newton}) metoda nie zbiega.
\begin{itemize}
    \item Jeżeli punkt startowy $x_0 \in (1, \infty)$, dla pewnych wartości takich jak $x_0=2.0$, czy $x_0=5.0$, metoda zbiega do miejsca zerowego. Jednak dla większych wartości, jak $x_0=10.0$, metoda nie zbiega i kończy się po osiągnięciu maksymalnej liczby iteracji.
\end{itemize}

%Metoda Newtona dla f2 z x0 = 10000.0:
%Miejsce zerowe: 10000.0, Wartosc funkcji w miejscu zerowym: 0.0, Iteracje: 0, Blad: 0
%Metoda Newtona dla f2 z x0 = 10.0:
%Miejsce zerowe: 15.455259647688075, Wartosc funkcji w miejscu zerowym: 2.9987642464534236e-6, Iteracje: 5, Blad: 0
%Metoda Newtona dla f2 z x0 = 5.0:
%Miejsce zerowe: 15.19428398343915, Wartosc funkcji w miejscu zerowym: 3.827247505782987e-6, Iteracje: 9, Blad: 0
%Metoda Newtona dla f2 z x0 = 2.0:
%Miejsce zerowe: 15.473297079378938, Wartosc funkcji w miejscu zerowym: 2.9485963624070995e-6, Iteracje: 11, Blad: 0
%Metoda Newtona dla f2 z x0 = 1.0:
%Miejsce zerowe: NaN, Wartosc funkcji w miejscu zerowym: NaN, Iteracje: 1000, Blad: 1
%Metoda Newtona dla f2 z x0 = 0.5:
%Miejsce zerowe: -3.0642493416461764e-7, Wartosc funkcji w miejscu zerowym: -3.0642502806087233e-7, Iteracje: 5, Blad: 0
%Metoda Newtona dla f2 z x0 = 0.0:
%Miejsce zerowe: 0.0, Wartosc funkcji w miejscu zerowym: 0.0, Iteracje: 0, Blad: 0
%Metoda Newtona dla f2 z x0 = -0.5:
%Miejsce zerowe: -3.0642493416461764e-7, Wartosc funkcji w miejscu zerowym: -3.0642502806087233e-7, Iteracje: 4, Blad: 0
%Metoda Newtona dla f2 z x0 = -1.0:
%Miejsce zerowe: -3.0642493416461764e-7, Wartosc funkcji w miejscu zerowym: -3.0642502806087233e-7, Iteracje: 5, Blad: 0
%Metoda Newtona dla f2 z x0 = -100.0:
%Miejsce zerowe: -4.356806237879908e-6, Wartosc funkcji w miejscu zerowym: -4.356825219681853e-6, Iteracje: 108, Blad: 0

\begin{table}[h!]
    \centering
    \begin{tabular}{|c|c|c|c|c|}
        \hline
        Punkt startowy & Przybliżone $r$ & Wartość $f_2(r)$ & Liczba iteracji & Kod Błędu \\
        \hline
        10000.0 & 10000.0 & 0.0 & 0 & 0 \\
        \hline
        10.0 & 15.455259647688075 & $2.9987642464534236\cdot 10^{-6}$ & 5 & 0 \\
        \hline
        5.0 & 15.19428398343915 & $3.827247505782987\cdot 10^{-6}$ & 9 & 0 \\
        \hline
        2.0 & 15.473297079378938 & $2.9485963624070995\cdot 10^{-6}$ & 11 & 0 \\
        \hline
        1.0 & NaN & NaN & 1 & 2 \\
        \hline
        0.5 & $-3.0642493416461764\cdot 10^{-7}$ & $-3.0642502806087233\cdot 10^{-7}$ & 5 & 0 \\
        \hline
        0.0 & 0.0 & 0.0 & 0 & 0 \\
        \hline
        -0.5 & $-3.0642493416461764\cdot 10^{-7}$ & $-3.0642502806087233\cdot 10^{-7}$ & 4 & 0 \\
        \hline
        -1.0 & $-3.0642493416461764\cdot 10^{-7}$ & $-3.0642502806087233\cdot 10^{-7}$ & 5 & 0 \\
        \hline
        -100.0 & $-4.356806237879908\cdot 10^{-6}$ & $-4.356825219681853\cdot 10^{-6}$ & 108 & 0 \\
        \hline
    \end{tabular}
    \caption{Wyniki działania metody Newtona na funkcji $f_2(x) = x e^{-x}$.}
    \label{tab:zad6_f2_newton}
\end{table}
Z wyników widać, że metoda Newtona może mieć problemy z zbieżnością jeżeli punkt startowy nie jest dobrze dobrany (lokalność). W przypadku funkcji $f_1$ im dalej od miejsca zerowego znajduje się punkt startowy, tym więcej iteracji potrzeba do znalezienia rozwiązania. Natomiast w przypadku funkcji $f_2$ punkty startowe większe od około 1 prowadzą do braku zbieżności (w przypadku $x_0 > 1$ wartości funkcji są efektywnie równe 0). Zatem:
\begin{itemize}
    \item jeżeli punkt startowy $x_0 \in (1, \infty)$, metoda nie zbiega do miejsca zerowego, kończy się po osiągnięciu wystarczająco małej wartości funkcji,
    \item jeżeli punkt startowy $x_0 = 1.0$, otrzymujemy błąd 2 (pochodna równa 0), faktycznie
        \[
            f_2'(1) = e^{-1} + (-1) e^{-1} = 0
        \]
\end{itemize}

\subsection{Metoda siecznych}

%x0 = 0.0, x1 = 2.0
%Miejsce zerowe: 1.0000017597132702, Wartosc funkcji w miejscu zerowym: -1.7597117218937086e-6, Iteracje: 6, Blad: 0
%x0 = -1.0, x1 = 1.0
%Miejsce zerowe: 1.0, Wartosc funkcji w miejscu zerowym: 0.0, Iteracje: 1, Blad: 0
%x0 = 1.5, x1 = 2.0
%Miejsce zerowe: 1.0000034269838276, Wartosc funkcji w miejscu zerowym: -3.4269779555229363e-6, Iteracje: 5, Blad: 0
%x0 = 2.5, x1 = 3.0
%Miejsce zerowe: 1.0000000980263084, Wartosc funkcji w miejscu zerowym: -9.80263036298723e-8, Iteracje: 11, Blad: 0
%x0 = -2.5, x1 = -3.0
%Miejsce zerowe: 0.9999998925933478, Wartosc funkcji w miejscu zerowym: 1.0740665801201033e-7, Iteracje: 10, Blad: 0
%x0 = 200.5, x1 = -300.0
%Miejsce zerowe: 200.5, Wartosc funkcji w miejscu zerowym: -1.0, Iteracje: 1, Blad: 0
%x0 = 200.5, x1 = 300.0
%Miejsce zerowe: NaN, Wartosc funkcji w miejscu zerowym: NaN, Iteracje: 1000, Blad: 1
\begin{table}[h!]
    \centering
    \begin{tabular}{|c|c|c|c|c|}
        \hline
        Punkty startowe & Przybliżone $r$ & Wartość $f_1(r)$ & Liczba iteracji & Kod Błędu \\
        \hline
        (0.0, 2.0) & 1.0000017597132702 & $-1.7597117218937086\cdot 10^{-6}$ & 6 & 0 \\
        \hline
        (-1.0, 1.0) & 1.0 & 0.0 & 1 & 0 \\
        \hline
        (1.5, 2.0) & 1.0000034269838276 & $-3.4269779555229363\cdot 10^{-6}$ & 5 & 0 \\
        \hline
        (2.5, 3.0) & 1.0000000980263084 & $-9.80263036298723\cdot 10^{-8}$ & 11 & 0 \\
        \hline
        (-2.5, -3.0) & 0.9999998925933478 & $1.0740665801201033\cdot 10^{-7}$ & 10 & 0 \\
        \hline
        (200.5, -300.0) & 200.5 & -1.0 & 1 & 0 \\
        \hline
        (200.5, 300.0) & NaN & NaN & 1000 & 1 \\
        \hline
    \end{tabular}
    \caption{Wyniki działania metody siecznych na funkcji $f_1(x) = e^{1-x} - 1$.}
    \label{tab:zad6_f1_secant}
\end{table}
Dla funkcji $f_1$ metoda siecznych działa dobrze dla różnych punktów startowych, jednak dla bardzo dużych wartości (ostatni wiersz tabeli \ref{tab:zad6_f1_secant}) metoda nie zbiega.



%x0 = 0.0, x1 = 2.0
%Miejsce zerowe: 0.0, Wartosc funkcji w miejscu zerowym: 0.0, Iteracje: 1, Blad: 0
%x0 = -1.0, x1 = 1.0
%Miejsce zerowe: 1.744165849924562e-8, Wartosc funkcji w miejscu zerowym: 1.7441658195034172e-8, Iteracje: 18, Blad: 0
%x0 = 1.5, x1 = 2.0
%Miejsce zerowe: 15.105971719963007, Wartosc funkcji w miejscu zerowym: 4.156315487897544e-6, Iteracje: 15, Blad: 0
%x0 = 2.5, x1 = 3.0
%Miejsce zerowe: 15.35695852170886, Wartosc funkcji w miejscu zerowym: 3.2874780854479134e-6, Iteracje: 15, Blad: 0
%x0 = -2.5, x1 = -3.0
%Miejsce zerowe: -2.839329775034345e-9, Wartosc funkcji w miejscu zerowym: -2.839329783096139e-9, Iteracje: 11, Blad: 0
%x0 = -200.5, x1 = -300.0
%Miejsce zerowe: -200.5, Wartosc funkcji w miejscu zerowym: -2.3886801389325308e89, Iteracje: 1, Blad: 0
%x0 = 200.5, x1 = 300.0
%Miejsce zerowe: 300.0, Wartosc funkcji w miejscu zerowym: 1.544460066723604e-128, Iteracje: 1, Blad: 0
\begin{table}[h!]
    \centering
    \begin{tabular}{|c|c|c|c|c|}
        \hline
        Punkty startowe & Przybliżone $r$ & Wartość $f_2(r)$ & Liczba iteracji & Kod Błędu \\
        \hline
        (0.0, 2.0) & 0.0 & 0.0 & 1 & 0 \\
        \hline
        (-1.0, 1.0) & $1.744165849924562\cdot 10^{-8}$ & $1.7441658195034172\cdot 10^{-8}$ & 18 & 0 \\
        \hline
        (1.5, 2.0) & 15.105971719963007 & $4.156315487897544\cdot 10^{-6}$ & 15 & 0 \\
        \hline
        (2.5, 3.0) & 15.35695852170886 & $3.2874780854479134\cdot 10^{-6}$ & 15 & 0 \\
        \hline
        (-2.5, -3.0) & $-2.839329775034345\cdot 10^{-9}$ & $-2.839329783096139\cdot 10^{-9}$ & 11 & 0 \\
        \hline
        (-200.5, -300.0) & -200.5 & $-2.3886801389325308\cdot 10^{89}$ & 1 & 0 \\
        \hline
        (200.5, 300.0) & 300.0 & $1.544460066723604\cdot 10^{-128}$ & 1 & 0 \\
        \hline
    \end{tabular}
    \caption{Wyniki działania metody siecznych na funkcji $f_2(x) = x e^{-x}$.}
    \label{tab:zad6_f2_secant}
\end{table}
Dla funkcji $f_2$ metoda siecznych również działa dobrze dla różnych punktów startowych, nawet dla bardzo dużych wartości.






\end{document}

