\documentclass[11pt,a4paper]{article}

\usepackage[T1]{fontenc}
\usepackage[utf8]{inputenc}
\usepackage[polish]{babel}
\usepackage{lmodern}
\usepackage[final]{microtype}

\usepackage[a4paper,margin=2.5cm]{geometry}
\usepackage{fancyhdr}
\pagestyle{fancy}
\fancyhf{} % czyść wszystko
\lhead{Notatki z matematyki}
\rhead{\leftmark}
\cfoot{\thepage}

\usepackage{amsmath,amssymb,amsthm,mathtools}
\usepackage{siunitx} % jeżeli potrzebne jednostki
\sisetup{locale = DE} % separator 1,23 (polski), można zmienić na locale=PL gdy dostępne
\usepackage{bm}       % pogrubione symbole

\usepackage[hidelinks]{hyperref}
\usepackage[nameinlink,capitalise,noabbrev]{cleveref}
\usepackage{csquotes}

\usepackage{enumitem}
\setlist{noitemsep,topsep=3pt}
\usepackage{xcolor}

\title{Obliczenia Naukowe -- Sprawozdanie Laboratoria 3.}
\author{Jakub Kogut}
\date{\today}

\begin{document}
\maketitle

\section{Wstęp}
Na 3. liscie zadań mamy za zadanie napisać oraz przetowstować różne metody rozwiązywania równiań $f(x) = 0$, gdzie $f$ jest zadaną funkcją rzeczywistą jednej zmiennej. 
\section{Metody iteracyjne}
\subsection{Metoda bisekcji}
Metoda bisekcji polega na podziale przedziału $[a,b]$, na którym funkcja $f$ zmienia znak, na pół i wyborze podprzedziału, na którym funkcja również zmienia znak. Proces ten iterujemy aż do uzyskania zadowalającej dokładności.

\end{document}

