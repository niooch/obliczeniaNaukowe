\documentclass[11pt,a4paper]{article}

\usepackage[T1]{fontenc}
\usepackage[utf8]{inputenc}
\usepackage[polish]{babel}
\usepackage{lmodern}
\usepackage[final]{microtype}
\usepackage{algorithm}
\usepackage{algpseudocode}


\usepackage[a4paper,margin=2.5cm]{geometry}
\usepackage{fancyhdr}
\pagestyle{fancy}
\fancyhf{} % czyść wszystko
\lhead{Sprawozdanie laboratoria 4.}
\rhead{\leftmark}
\cfoot{\thepage}

\usepackage{amsmath,amssymb,amsthm,mathtools}
\usepackage{siunitx} % jeżeli potrzebne jednostki
\sisetup{locale = PL} % separator 1,23 (polski), można zmienić na locale=PL gdy dostępne
\usepackage{bm}       % pogrubione symbole

\usepackage[hidelinks]{hyperref}
\usepackage[nameinlink,capitalise,noabbrev]{cleveref}
\usepackage{csquotes}

\usepackage{enumitem}
\setlist{noitemsep,topsep=3pt}
\usepackage{xcolor}

\title{Obliczenia Naukowe -- Sprawozdanie Laboratoria 4.}
\author{Jakub Kogut}
\date{\today}

\begin{document}
\maketitle

\section{Wstęp}
Na liście pojawia się problem interpolacji funkcji; polega ona na znalezieniu funkcji wielomianowej, która przechodzi przez zadane punkty i w ``dobry sposób'' przybliża funkcję oryginalną.\\
Dokładniej dla zadanych $(x_i, f(x_i))$ \textit{węzłów interpolacji}, gdzie $i = 0, 1, \ldots, n$ chcemy znaleźć wielomian $p_n(x)$ stopnia co najwyżej $n$, taki że:
\begin{equation}
    p_n(x_i) = f(x_i), \quad i = 0, 1, \ldots, n.
\end{equation}
Również celem jest zapisanie go w postacji Newtona:
\begin{equation}
    p_n(x) = \sum_{i=0}^{n} f[x_0, x_1, \ldots, x_i] \prod_{j=0}^{i-1} (x - x_j),
\end{equation}
gdzie $f[x_0, x_1, \ldots, x_i]$ są \textit{ilorazami różnicowymi} funkcji $f$ w punktach $x_0, x_1, \ldots, x_i$.

\subsection{Motywacja}
W tej sekcji wyjaśnie dlaczego stosujemy postać Newtona do zapisu wielomianu interpolacyjnego.
\subsubsection{Dlaczego nie postać naturalna?}
Postać naturalna wielomianu to:
\begin{equation}
    p_n(x) = a_0 + a_1 x + a_2 x^2 + \ldots + a_n x^n.
\end{equation}
Chociaż jest to najprostsza postać, to ma kilka wad:
\begin{itemize}
    \item Trudność w obliczaniu współczynników $a_i$ z danych punktów (rozwiązanie równania z macierzą Vandermonde'a, która jest źle uwarunkowana).
    \item Niestabilność numeryczna dla dużych $n$
\end{itemize}
\subsubsection{Dlaczego postać Newtona?}
Postać Newtona jest zbudowana na innej bazie:
\begin{equation}\begin{split}
    B &= \left\{ \prod_{j=0}^{k-1} (x - x_j) : k = 0, 1, \ldots, n \right\} \\
      &= \left\{1, (x - x_0), (x - x_0)(x - x_1), \ldots, (x - x_0)(x - x_1) \cdots (x - x_{n-1})\right\}.
\end{split}\end{equation}
Wtedy $p_n(x)$ można zapisać jako:
\begin{equation}
    p_n(x) = \sum_{i=0}^{n} c_i \prod_{j=0}^{i-1} (x - x_j),
\end{equation}
gdzie współczynniki $c_i$ są równe ilorazom różnicowym:
\begin{equation}
    c_i = f[x_0, x_1, \ldots, x_i].
\end{equation}
Postać Newtona ma kilka zalet:
\begin{itemize}
    \item Łatwość obliczania współczynników za pomocą ilorazu różnicowego (zadanie 1.)
    \item Łatwe obliczanie wartości wielomianu w danym punkcie za pomocą schematu Hornera (zadanie 2.)
    \item Stabilność numeryczna (mniejsze błędy przy dużych $n$)
\end{itemize}
\section{Implementacja}
Implementacja algorytmów znajduje się w pliku \texttt{interpolacja.jl}. Poniżej znajduje się opis zadań.
\section{Zadanie 1.}
W zadaniu 1. należy zaimplementować funkcję obliczającą ilorazy różnicowe dla zadanych węzłów interpolacji i wartości funkcji w tych punktach.\\
Poniżej znajduje się pseudokod algorytmu:
\begin{algorithm}
\caption{ilorazyRoznicowe}
\begin{algorithmic}[1]
\Function{ilorazyRoznicowe}{$x, f$}
    \State $n \gets \text{length}(x)$
    \State $F \gets \text{array of size } n$
    \For{$i \gets 0$ to $n-1$}
        \State $F[i] \gets f[i]$
    \EndFor
    \For{$j \gets 1$ to $n-1$}
        \For{$i \gets n-1$ downto $j$}
            \State $F[i] \gets \dfrac{F[i] - F[i-1]}{x[i] - x[i-j]}$
        \EndFor
    \EndFor
    \State \Return $F$
\EndFunction
\end{algorithmic}
\end{algorithm}


\end{document}

